
%%%%%%%%%%%%%%%%%%%%%%%%%%%%%%%%%%%%%%%%%
% University Assignment Title Page
% LaTeX Template
% Version 1.0 (27/12/12)
%
% This template has been downloaded from:
% http://www.LaTeXTemplates.com
%
% Original author:
% WikiBooks (http://en.wikibooks.org/wiki/LaTeX/Title_Creation)
%
% License:
% CC BY-NC-SA 3.0 (http://creativecommons.org/licenses/by-nc-sa/3.0/)
%
% Instructions for using this template:
% This title page is capable of being compiled as is. This is not useful for
% including it in another document. To do this, you have two options:
%
% 1) Copy/paste everything between \begin{document} and \end{document}
% starting at \begin{titlepage} and paste this into another LaTeX file where you
% want your title page.
% OR
% 2) Remove everything outside the \begin{titlepage} and \end{titlepage} and
% move this file to the same directory as the LaTeX file you wish to add it to.
% Then add \input{./title_page_1.tex} to your LaTeX file where you want your
% title page.
%
%%%%%%%%%%%%%%%%%%%%%%%%%%%%%%%%%%%%%%%%%
%\title{Title page with logo}
%----------------------------------------------------------------------------------------
%	PACKAGES AND OTHER DOCUMENT CONFIGURATIONS
%----------------------------------------------------------------------------------------

\documentclass[a4paper]{report}

%====================== PACKAGES ======================
\usepackage{bbold}
\usepackage{soul}               %Pour souligner
\usepackage{dsfont}
\usepackage[french]{babel}
\usepackage[utf8]{inputenc}		% Si encodage du document en utf8
\usepackage[T1]{fontenc}
% Pour avoir le document en français

%\usepackage[utf8x]{inputenc}	% Encodage du document
\usepackage{float}
\usepackage{subcaption}				% Pour gérer les positionnement d'images
\usepackage{graphicx}
\usepackage{amsmath}
\usepackage{amsfonts}
\usepackage{mathrsfs}			% Pour les lettres calligraphiques équation
\usepackage[colorinlistoftodos]{todonotes}
\usepackage{url}				% Pour faire des hyperliens vers le web
\usepackage{xcolor}
% pour les informations sur un document compilé en PDF et les liens externes / internes
\usepackage{hyperref}			% Pour faire des hyperliens
\usepackage{array}				% Pour faire des tableaux
\usepackage{tabularx}
% pour utiliser 		% floatbarrier
%\usepackage{placeins}
%\usepackage{floatrow}
\usepackage{setspace}			% Espacement entre les lignes
\usepackage{abstract}			% Modifier la mise en page de l'abstract
\usepackage[T1]{fontenc}		% Police et mise en page (marges) du document
\usepackage[top=2cm, bottom=2cm, left=2cm, right=2cm]{geometry}
\usepackage{pdfpages}			% pour inclures des pdf comme des images
%\usepackage{subfig}				% Pour les galerie d'images
\usepackage{listings}			% pour inclure du code dans le doc
\usepackage{soul}				% Pour surligner
\usepackage{enumitem}

\usepackage{tcolorbox}

\sethlcolor{grisclair}
\definecolor{darkgreen}{RGB}{0,100,0}
\definecolor{MonBleu}{RGB}{84, 114,174}
\definecolor{MonPrune}{RGB}{150,25,49}
\definecolor{MonMarron}{RGB}{165, 38, 10}
 \usepackage{tikz}
 \usepackage{schemabloc}
 \usepackage{afterpage}
\usepackage{float}

\usepackage{listings}
% ----------------------------------------------------------------

\definecolor{mGreen}{rgb}{0,0.6,0}
\definecolor{mGray}{rgb}{0.5,0.5,0.5}
\definecolor{mPurple}{rgb}{0.58,0,0.82}
\definecolor{backgroundColour}{rgb}{0.95,0.95,0.92}
\definecolor{mygreen}{RGB}{28,172,0} % color values Red, Green, Blue
\definecolor{mylilas}{RGB}{170,55,241}

\lstdefinestyle{myStyle}{language=matlab,
    backgroundcolor=\color{backgroundColour},
    commentstyle=\color{mGreen},
    keywordstyle=\color{magenta},
    numberstyle=\tiny\color{mGray},
    stringstyle=\color{mPurple},
    basicstyle=\footnotesize,
    breakatwhitespace=false,
    breaklines=true,
    captionpos=b,
    keepspaces=true,
    numbers=left,
    numbersep=5pt,
    showspaces=false,
    showstringspaces=false,
    showtabs=false,
    tabsize=2,
    literate=
    {á}{{\'a}}1 {é}{{\'e}}1 {í}{{\'i}}1 {ó}{{\'o}}1 {ú}{{\'u}}1
    {Á}{{\'A}}1 {É}{{\'E}}1 {Í}{{\'I}}1 {Ó}{{\'O}}1 {Ú}{{\'U}}1
    {à}{{\`a}}1 {è}{{\`e}}1 {ì}{{\`i}}1 {ò}{{\`o}}1 {ù}{{\`u}}1
    {À}{{\`A}}1 {È}{{\'E}}1 {Ì}{{\`I}}1 {Ò}{{\`O}}1 {Ù}{{\`U}}1
    {ä}{{\"a}}1 {ë}{{\"e}}1 {ï}{{\"i}}1 {ö}{{\"o}}1 {ü}{{\"u}}1
    {Ä}{{\"A}}1 {Ë}{{\"E}}1 {Ï}{{\"I}}1 {Ö}{{\"O}}1 {Ü}{{\"U}}1
    {â}{{\^a}}1 {ê}{{\^e}}1 {î}{{\^i}}1 {ô}{{\^o}}1 {û}{{\^u}}1
    {Â}{{\^A}}1 {Ê}{{\^E}}1 {Î}{{\^I}}1 {Ô}{{\^O}}1 {Û}{{\^U}}1
    {œ}{{\oe}}1 {Œ}{{\OE}}1 {æ}{{\ae}}1 {Æ}{{\AE}}1 {ß}{{\ss}}1
    {ű}{{\H{u}}}1 {Ű}{{\H{U}}}1 {ő}{{\H{o}}}1 {Ő}{{\H{O}}}1
    {ç}{{\c c}}1 {Ç}{{\c C}}1 {ø}{{\o}}1 {å}{{\r a}}1 {Å}{{\r A}}1
    {€}{{\euro}}1 {£}{{\pounds}}1 {«}{{\guillemotleft}}1
    {»}{{\guillemotright}}1 {ñ}{{\~n}}1 {Ñ}{{\~N}}1 {¿}{{?`}}1 {\ \ }{{\ }}1}

\lstdefinestyle{matlab}{language=matlab,
    backgroundcolor=\color{backgroundColour},
    keywordstyle=\color{blue},%
    morekeywords=[2]{1}, keywordstyle=[2]{\color{black}},
    identifierstyle=\color{black},%
    stringstyle=\color{mylilas},
    commentstyle=\color{mygreen},%
    basicstyle=\footnotesize,
    breakatwhitespace=false,
    breaklines=true,
    captionpos=b,
    keepspaces=true,
    numbers=left,
    numbersep=5pt,
    showspaces=false,
    showstringspaces=false,
    showtabs=false,
    tabsize=2,
    literate=
    {á}{{\'a}}1 {é}{{\'e}}1 {í}{{\'i}}1 {ó}{{\'o}}1 {ú}{{\'u}}1
    {Á}{{\'A}}1 {É}{{\'E}}1 {Í}{{\'I}}1 {Ó}{{\'O}}1 {Ú}{{\'U}}1
    {à}{{\`a}}1 {è}{{\`e}}1 {ì}{{\`i}}1 {ò}{{\`o}}1 {ù}{{\`u}}1
    {À}{{\`A}}1 {È}{{\'E}}1 {Ì}{{\`I}}1 {Ò}{{\`O}}1 {Ù}{{\`U}}1
    {ä}{{\"a}}1 {ë}{{\"e}}1 {ï}{{\"i}}1 {ö}{{\"o}}1 {ü}{{\"u}}1
    {Ä}{{\"A}}1 {Ë}{{\"E}}1 {Ï}{{\"I}}1 {Ö}{{\"O}}1 {Ü}{{\"U}}1
    {â}{{\^a}}1 {ê}{{\^e}}1 {î}{{\^i}}1 {ô}{{\^o}}1 {û}{{\^u}}1
    {Â}{{\^A}}1 {Ê}{{\^E}}1 {Î}{{\^I}}1 {Ô}{{\^O}}1 {Û}{{\^U}}1
    {œ}{{\oe}}1 {Œ}{{\OE}}1 {æ}{{\ae}}1 {Æ}{{\AE}}1 {ß}{{\ss}}1
    {ű}{{\H{u}}}1 {Ű}{{\H{U}}}1 {ő}{{\H{o}}}1 {Ő}{{\H{O}}}1
    {ç}{{\c c}}1 {Ç}{{\c C}}1 {ø}{{\o}}1 {å}{{\r a}}1 {Å}{{\r A}}1
    {€}{{\euro}}1 {£}{{\pounds}}1 {«}{{\guillemotleft}}1
    {»}{{\guillemotright}}1 {ñ}{{\~n}}1 {Ñ}{{\~N}}1 {¿}{{?`}}1 {\ \ }{{\ }}1}
%----------------------------------------------------


% Environnement théorème
\usepackage[framemethod=TikZ]{mdframed}
\newcounter{theo}[section]\setcounter{theo}{0}
\renewcommand{\thetheo}{\arabic{section}.\arabic{theo}} % Modif 17/02/2019
%\newenvironment{name}[args]{begin_def}{end_def}

\newenvironment{theo}[2][]{%
    \refstepcounter{theo}
        \ifstrempty{#1}%
        % if condition (without title)
        {\mdfsetup{%
            frametitle={%
                \tikz[baseline=(current bounding box.east),outer sep=0pt]
                \node[anchor=east,rectangle,fill=red!20]
                {\strut};}
            }%
        % else condition (with title)
        }{\mdfsetup{%
            frametitle={%
                \tikz[baseline=(current bounding box.east),outer sep=0pt]
                \node[anchor=east,rectangle,fill=red!20]
                {\strut ~#1};}%
            }%
        }%
        % Both conditions
        \mdfsetup{%
            innertopmargin=10pt,linecolor=red!20,%
            linewidth=2pt,topline=true,%
            frametitleaboveskip=\dimexpr-\ht\strutbox\relax%
        }
\begin{mdframed}[]\relax}{%
\end{mdframed}}



%\pgfversion

%====================== INFORMATION ET REGLES ======================

%rajouter les numérotation pour les \paragraphe et \subparagraphe
\setcounter{secnumdepth}{4}
\setcounter{tocdepth}{4}

\newcounter{cpt1}						% Compteur pour les n° de ligne dans les prog de l'annexe1
\newcommand\increm{\arabic{cpt1}\addtocounter{cpt1}{1}}
%initialisation de l'intégrateur de language C

\usepackage[explicit]{titlesec}

%%%%% Ce que j'ai ajouté
% Bjornstrup - Rejne - Conny - Glenn - Sonny - Lenny - Bjarne
\usepackage[Conny]{fncychap}

\AddThinSpaceBeforeFootnotes
\FrenchFootnotes

\begin{document}
% \newcommand{\HRule}{\rule{\linewidth}{0.5mm}}

%page de garde
\input{./page_de_garde/page_de_garde.tex}
%page blanche
\newpage

~
\thispagestyle{empty}
\setcounter{page}{0}

%ne pas numéroter le sommaire
\newpage
\tableofcontents
\thispagestyle{empty}
\setcounter{page}{1}
%espacement entre les lignes d'un tableau
\renewcommand{\arraystretch}{1.5}

%====================== INCLUSION DES PARTIES ======================
%
%~
\thispagestyle{empty}
%recommencer la numérotation des pages à "1"
\setcounter{page}{0}

\begin{table}[h]
    \centering
    \begin{tabularx}{0.8\textwidth} {
  | >{\raggedright\arraybackslash}X
  | >{\raggedright\arraybackslash}X
  | >{\raggedright\arraybackslash}X | }
  \hline
  Acteur & Système\\
  \hline
  1: Le client présente son badge & 2: Le système identifie le client \\
   & 3: Le système fait clignoter le voyant charge en vert pendant 8s \\
  4: Le client à une minute pour appuyer sur le bouton charge & 5: Le système éteint le voyant disponible \\
  6: Le client retire sa carte & 7: "Lancement UC Charger Batterie" Le système déclenche la charge du vehicule \\
   & 8: Le système met le voyant dispo à Vert.\\
  \hline
\end{tabularx}
    \caption{Scénario UC Recharger Batterie}
    \label{tab:my_label}
\end{table}

\begin{table}[h]
    \centering
    \begin{tabularx}{0.8\textwidth} {
  | >{\raggedright\arraybackslash}X
  | >{\raggedright\arraybackslash}X
  | >{\raggedright\arraybackslash}X | }
  \hline
  Acteur & Système\\
  \hline
  1: Le client présente son badge & 2: Le système identifie le client \\
   & 3: Le système déverrouille la trappe \\
  4: Le client débranche la prise & 5: Le système éteint le voyant prise \\
   & 6: Le système met le voyant dispo à Vert
  \hline
\end{tabularx}
    \caption{Scénario UC Reprendre Vehicule}
    \label{tab:my_label}
\end{table}

\begin{table}[h]
    \centering
    \begin{tabularx}{0.8\textwidth} {
  | >{\raggedright\arraybackslash}X
  | >{\raggedright\arraybackslash}X
  | >{\raggedright\arraybackslash}X | }
  \hline
  Acteur & Système\\
  \hline
  1: Le client présente son badge & 2: Le système ne peut authentifier le client \\
   & 3: Le système fait clignoter la led défaut pendant 8s \\
  4: Le client retire sa carte &\\
  \hline
\end{tabularx}
    \caption{Scénario UC Reprendre Vehicule Variante}
    \label{tab:my_label}
\end{table}





\end{document}
